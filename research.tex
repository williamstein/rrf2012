\documentclass[11pt]{article}
\usepackage{url}
\usepackage{graphicx}
\usepackage{pdfpages}
\usepackage{hyperref}
\include{macros}
\newcommand{\thisdocument}{RRF: Proposed Research}

\begin{document}
\section{Introduction and Rationale}

%[[Provide brief critical review of the pertinent literature, theoretical
%background, and justification for the proposed research. Describe any
%results already achieved, including publications.]]

Most of the PI's published research work is in number theory, which is
an area of pure mathematics.  The PI embarked on a new and risky path
at Harvard in 2005, when he started the Sage mathematical software
project, which has grown substantially since he was recruited by UW in
2006.  There are now around 500 Sage developers, over 10,000 downloads
of Sage per month, over 55,000 unique visitors each month to the
\url{http://sagemath.org} website, and many publications make use of
Sage (see \url{http://www.sagemath.org/library-publications.html}).
Sage is a very large free
self-contained program for doing sophisticated mathematics, which you
can download and run on your computer.  It incorporates nearly 100
other programs, and already has interfaces to all other popular math
software systems.  The goal of the Sage project is to create a viable
free open source alternative to the commercial systems Maple,
Mathematica, Matlab and Magma.  The PI has published two articles
about Sage (see the citations document).

The Sage notebook, whose development was partly supported by the RRF
program in 2009, is the primary user interface to Sage.  It was
designed mainly for use by a relatively small number of simultaneous
users, but there is a public free version that anybody can quickly
start using at \url{http://sagenb.org}.  Sage is software for doing
number crunching, so it is resource intensive.  The
\url{http://sagenb.org} website runs on one powerful computer that the
PI purchased in {\bf 2008} and it has around 100,000 user accounts!
There are often well over 50 people attempting to use
\url{http://sagenb.org} at any given time, which can be frustrating.
The core design of sagenb is not-scalable to multiple nodes, since it
doesn't even use a database, and rewriting it to use one has proved
extremely challenging. Also, there is currently no way for users
to pay a small fee to rent more available resources.

\begin{center}
\begin{figure}[ht]
\hfill
\includegraphics[height=0.2\textheight]{nb2}
\hfill
\includegraphics[height=0.2\textheight]{nb1}
\hfill
\mbox{}
\caption{Screenshots Showing the Sage Notebook\label{fig:sagenb}}
\end{figure}
\end{center}

\noindent{}{\bf Goal:} {\em Create a web application (based on Sage)
  called Salvus, which is designed from the start to be secure, highly
  available, horizontally scalable and able to robustly handle up to
  1,000 simultaneous users running active computations (far more users
  could be connected).  Provide both free and non-free accounts, with
  revenue from the non-free subscriptions supporting maintenance and
  purchase of hardware, and the development of Sage.}

Instead of teachers and researchers having to pay to buy expensive
commercial mathematical software and install it on their computers, or
be frustrated with the current Sage notebook, Salvus would provide
them the option to {\em reliably} and efficiently use Sage over the
web and confidently get their students to use it.  At many
institutions, purchasing computer software---especially mathematical
software---is a significant burden, and Sage has helped address this
problem. The burden of installing, maintaining, and upgrading Sage
remains, and the Sage notebook has not solved that problem because it
is too slow and unreliable.  This project has the potential to have a
profound impact on education at all levels.

There are several commercial web-based software development
environments such as \url{https://c9.io/} and
\url{http://www.heroku.com/}, which have some similarities to what the
PI plans to implement with Salvus.  However, Salvus will be highly
interactive and much more focused on mathematical algorithms,
education, and research than on deploying generic web applications.

\section{Objectives}
%What is the project designed to accomplish?

The PI spent much of Summer 2012 on the foundations of Salvus, and
will make a limited-functionality version available soon (see
\url{https://salv.us}).  He intends to have a version up and running
with comparable functionality to the existing Sage Notebook by March
2013, and build a core subscriber base of at least 2500 users by
August 2013; work during Winter 2013 is being funded by the remainder
of his startup money, and during some of Summer 2013 by an NSF grant.

This RRF would allow the PI to work
fulltime through late December 2013 on the next stage
of the Salvus project, to address these specific goals:
%. The specific goals for the second stage of
%development will address some significant {\em functionality} that is
%missing from the current Sage Notebook and Salvus.

\begin{enumerate}
\item Market an enterprise version of Salvus that customers can
  install on their own internal secure network.
  Similarly, support university and course wide site licenses for
  Salvus.
\item Greatly improve support for using software other than Sage
  through the Salvus interface, including Octave, Scilab, R (included
  in Sage), Matlab, Mathematica, Maple, and Magma. This will draw in
  users from the sciences.
\item Implement editors for at least the following document types:
  command line, \LaTeX{} document, Mathcad style free-form document,
  Python and Sage files, and for fast interactive 2d and 3d graphics
  using modern web standard such as HTML5 canvas and WebGL.
\item Integrate with many popular cloud storage and repository hosting
  services, including Dropbox, Google Drive, Github, Google Code, and
  Bitbucket.
\item Make it so users can write code that
  gets included with Sage entirely using Salvus, so they can
  contribute to Sage without having to install Sage.
\end{enumerate}


\section{Procedure}

% With what methods, materials, or tools will the objectives be met?
% If access to a particular location or institution is required for
% research or data collection, state whether permission has been
% obtained.

\subsection{Computer Hardware}
A single server can serve static pages to 1,000 simultaneous users
reasonably quickly.  The real challenge is that when users evaluate
mathematical expressions, such as computation of a difficult symbolic
integral, they are potentially using substantial CPU
resources. Moreover, we must keep a persistent isolated Sage process
running for the duration of their session.  We will thus require a
large number of computers that have the ability to run a Sage session.
For high availability, there must be machines in many physical
locations.

Purchasing access to computers located at multiple data centers is
critical to this proposal, given the dual goals of high availability
(even when a node or whole data center goes down) and scalability
(1,000 {\em simultaneous} users).  Use of Sage is extremely CPU
intensive, and based on extensive PI experience\footnote{During Summer
  2012, the PI ran a 2-week ``SIMUW'' workshop at UW, with about 30
  high school students hammering a Sage notebook server.}, which
includes dozens of hands on tutorials over 5 years using the Sage
notebook, the PI estimates that appropriately implemented, $n$-Ghz
of compute power and $n$-gigabytes RAM translates to a good experience
for roughly $2n$ simultaneous users.  The PI is rolling out the
following compute acquisition plan over the next 6 months:

\begin{enumerate}
\item {\bf Padelford Hall:} 5 Dell R415's each with 64GB RAM
  and 16 3Ghz cores (cost of \$12,500K; no hosting costs) --
  480 simultaneous users. 

\item {\bf UW Tower (Colocation):} 5 Dell R415's with
  64GB RAM and 16 3Ghz cores (cost of \$12,500K; yearly
  cost of \$1200) -- 480 simultaneous users. 

\item {\bf Google App Engine:} the PI was awarded \$60K in Google App
  Engine credits via a competitive application procedure.  Google App
  Engine is a restricted application hosting environment, which can't
  yet run Sage; the PI is working with Google to determine how best to
  make use of these resources.

\item {\bf Servedby.net cloud computing
  provider}\footnote{\url{http://servedby.net/} is owned and run by a
  recent UW undergraduate.}: 8 Linux 4R instances (about \$15K/year)
  -- 64 simultaneous users.

\item {\bf Amazon.com EC2 could hosting:} 8 m1.large spot instances
  (four 1 Ghz cores each).  One bids on these; they are cheaper than
  guaranteed instances, but are not always available (about
  \$10K/year) -- 64 simultaneous users.

\item {\bf UW backup:} 500GB of UW's Tivoli tape backup
  at two sites (\$365/year).

\end{enumerate}

The PI is focusing initially on hardware hosted at UW, since power,
network bandwidth and hosting are highly subsidized by other UW
activities.  (Also, spare cycles on this hardware will be used for
mathematics research projects while the user base for Salvus grows.)

When a node (or even a whole data center) go down, the site will
continue to give users full access to their data, and they will be
able to perform computations.  However, users will share less compute
resources, so running computations may be slower.  Non-paying users
will have a lower priority for resources.

If the number of paying users grows so that sufficient revenue is
being generated from subscriptions, it will be easy to increase cloud
computing resources to meet demands.  Moreover, cloud
hosting provides the option to have servers located around the world,
and to dynamically grow and shrink in response to demand.

\subsection{Computer Software}

To support fault tolerance, the PI has designed Salvus so that it has
no single (or even double!) points of failure.  The underlying
database is {\bf Cassandra} (see \url{http://www.datastax.com/}),
which is designed so that it continues to function well even if
several nodes (or even a whole data center) fails.  Cassandra is
heavily used by Netflix, Disney, and many other companies.

For internode communications security, we use the encrypted
fault-tolerant virtual private network software {\bf tinc} to connect
all machines together (tinc is a P2P VPN with no single
point of failure, in contrast to openVPN).  For communications
security between our cluster(s) and end users, we use {\bf stunnel}
and SSL, with each data center having one stunnel node and DNS pointed
at all of the them.  For HTTP/1.1 load balancing we use {\bf haproxy}.
We use the {\bf Tornado} web server, which excels at serving a large
number of standing connections.  We serve static content using {\bf
  Nginx}.  The heavy lifting is done by nodes
running {\bf Sage}.


\section{Time Schedule}
% Provide a schedule showing how the proposed research can be
% accomplished during the desired support period. The support period
% is normally limited to one year. A no-cost extension of up to one
% year may be granted if requested and adequately justified.

The PI proposes to work full time for three months supported by the RRF:

\begin{enumerate}
\item {\bf October, 2013:} Package and test a standalone enterprise
  version of Salvus.  Work with lawyers at C4C to nail down licensing
  and copyright issues. The PI has existing longterm relationships
  with two large potential customers.  Improve support
  for R, Matlab, Magma, etc., which is of interest to such customers,
  and support university and course wide site licenses for Salvus.

\item {\bf November, 2013:} Improve support for a wide range of
  filetypes, while integrating Salvus with popular cloud storage and
  hosting services.
\item {\bf December, 2013:} Building on support for repository hosting
  from the previous month, make it possible for people to write and
  modify the code that gets included with Sage entirely using Salvus.
  This will indirectly contribute to the ultimate goal of improving
  the core capabilities of Sage.
\end{enumerate}


\section{Need for RRF Support}
% What other efforts have been made to find support for the project?
% How could the results of the work lead to further outside funding or
% commercial applications? How does this project address the mission
% of the Royalty Research Fund? For RRF Scholar applicants, provide
% documentation of teaching load (quarter, course number, title, and
% credits) in this section.

Though the PI has obtained funding for the Sage project from
Microsoft, Google, the National Science Foundation, prize money,
private donations, and contracts for Sage development, this has mostly
been limited seed money.  The PI is the main architect and programmer
of Salvus, and currently has no source of funding to pay his own
salary during the next academic year.  Salvus subscriptions may
eventually provide such funding, but this RRF would address the
critical period before the subscriber base grows sufficiently.


As explained above, the goals of the current proposal are to greatly enhance Sage's level
of {\em robust availability} and provide a sustainable UW-based
revenue stream to support Sage development.  If successful, this will
have a huge and direct impact both on development of Sage and the
community of Sage users.  This project thus directly addresses the
mission of the Royalty Research Fund by providing a unique opportunity
to increase the PI's competitiveness for subsequent funding via
individual user and site-wide subscriptions to Salvus.

The last RRF that the PI received in 2009 helped make Sage attractive
to the {\em educational market}, which is a huge area of potential
funding that Sage had not had success in yet.  The PI subsequently
received a substantial CCLI Type 2 grant (see
\url{http://utmost.aimath.org/}) from the educational part of NSF.

\subsection{RRF Scholar Application: documentation of teaching load}
The math teaching load is about 4.5 quarter courses per year (we have
a more complicated internal point scheme).  The PI is requesting
replacement salary for 2 Calculus courses (e.g., MATH 124: {\em
  Calculus with Analytic Geometry I}, 5 credits).

\subsection{C4C Letter of Support}\label{sec:letter}
This is a letter of support from the UW Center for Commercialization.

\includepdf[pages=-]{letter_from_c4c}

\end{document}
