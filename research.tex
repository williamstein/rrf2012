\documentclass[11pt]{article}
\usepackage{url}
\usepackage{graphicx}
\usepackage{amsmath}%
\usepackage{amsfonts}%
\usepackage{amssymb}%
\usepackage{amsthm}%

\hoffset=-.04\textwidth%
\textwidth=1.08\textwidth%
\voffset=-0.03\textheight
\textheight=1.06\textheight
%\voffset=0.5cm%
%\textheight=1.08\textheight%

\newcommand{\C}{\mathbf{C}}%
\newcommand{\F}{\mathbf{F}}%
\newcommand{\Q}{\mathbf{Q}}%
\newcommand{\Qbar}{\overline{\Q}}%
\newcommand{\Z}{\mathbf{Z}}%
\newcommand{\R}{\mathbf{R}}%
\newcommand{\T}{\mathbf{T}}%
\renewcommand{\H}{\mathrm{H}}%

\newcommand{\eps}{\varepsilon}%
\newcommand{\con}{\equiv}%
\newcommand{\isom}{\cong}%
\newcommand{\rhobar}{\overline{\rho}}
\newcommand{\tensor}{\otimes}


% ---- SHA ----
\DeclareFontEncoding{OT2}{}{} % to enable usage of cyrillic fonts
  \newcommand{\textcyr}[1]{%
    {\fontencoding{OT2}\fontfamily{wncyr}\fontseries{m}\fontshape{n}%
     \selectfont #1}}%
\newcommand{\Sha}{{\mbox{\textcyr{Sh}}}}%
\newcommand{\tor}{\mbox{\scriptsize\rm tor}}

\newcommand{\myname}{William Stein}
\newcommand{\phone}{{\sf (206) 419-0925}}
\newcommand{\email}{{\sf wstein@uw.edu}}
\newcommand{\www}{{\sf http://wstein.org}}
\newcommand{\address}{}
\usepackage{fancyhdr,ifthen}
\pagestyle{fancy}
\cfoot{\thepage}  % no footers (in pagestyle fancy)
% running left heading
\lhead{\bfseries\LARGE\em \noindent{}\hspace{-.2em}\myname{}
        \hfill \thisdocument\vspace{-.2ex}\\}
% running right heading
%\newcommand{\spc}{1.31em}
\newcommand{\spc}{1em}

%\rhead{\em {\small{\phone{}}} \hfill $\cdot$\hfill
% \email{} \hfill $\cdot$\hfill \www{}}

\rhead{\em {\small{\phone{}}} \hfill \email{} \hfill \www{}}

\setlength{\headheight}{5ex}
\newcommand{\mainhead}[1]{\begin{center}{\Large \bf #1}\end{center}}
\newcommand{\head}[1]{\vspace{1.5ex}\par\noindent{\large \bf #1}\par\noindent}
\newcommand{\subhead}[1]{\vspace{2ex}\par\noindent{\sl #1}\vspace{1ex}\par\noindent{}}
\newcommand{\ptitle}{\sl}

\newcommand{\hra}{\hookrightarrow}


%%%% Theoremstyles
\theoremstyle{plain}
\newtheorem{theorem}{Theorem}[section]
\newtheorem{proposition}[theorem]{Proposition}
\newtheorem{corollary}[theorem]{Corollary}
\newtheorem{claim}[theorem]{Claim}
\newtheorem{lemma}[theorem]{Lemma}
\newtheorem{conjecture}[theorem]{Conjecture}

\theoremstyle{definition}
\newtheorem{definition}[theorem]{Definition}
\newtheorem{algorithm}[theorem]{Algorithm}
\newtheorem{question}[theorem]{Question}
\newtheorem{problem}[theorem]{Problem}
\newtheorem{goal}[theorem]{Goal}

\theoremstyle{remark}
\newtheorem{remark}[theorem]{Remark}
\newtheorem{remarks}[theorem]{Remarks}
\newtheorem{example}[theorem]{Example}
\newtheorem{exercise}[theorem]{Exercise}

\DeclareMathOperator{\End}{End}%
\DeclareMathOperator{\Tr}{Tr}%
\DeclareMathOperator{\Res}{Res}%
\DeclareMathOperator{\res}{res}%
\DeclareMathOperator{\BSD}{BSD}%
\DeclareMathOperator{\Gal}{Gal}%
\DeclareMathOperator{\GL}{GL}%
\DeclareMathOperator{\Aut}{Aut}%
\DeclareMathOperator{\Reg}{Reg}%
\DeclareMathOperator{\Vis}{Vis}%
\DeclareMathOperator{\Ker}{Ker}%
\DeclareMathOperator{\Coker}{Coker}%
\DeclareMathOperator{\Sel}{Sel}%
\DeclareMathOperator{\ord}{ord}%
\DeclareMathOperator{\new}{new}%
\DeclareMathOperator{\an}{an}%


\newcommand{\thisdocument}{RRF: Proposed Research}

\begin{document}
\section{Introduction and Rationale}

%[[Provide brief critical review of the pertinent literature, theoretical
%background, and justification for the proposed research. Describe any
%results already achieved, including publications.]]

Most of the PI's (principal investigator's) research has been in
number theory, which is an area of pure mathematics.  The PI embarked
on a new and risky research path at Harvard in 2005, when he started
the Sage mathematical software project, which has greatly grown since
he was recruited by UW in 2006.  There are now over 150 Sage
developers, around 2,000 downloads of Sage per month, and an average
of over 1,500 unique visitors to the \url{http://sagemath.org} website
each day.  This group of users and developers is guided by a single
clear vision for Sage:

\begin{quote} {\em Create a viable free open source alternative
    to the commercial systems Maple, Mathematica, Matlab and Magma.}
\end{quote}

Sage can be used to study general and advanced, pure and applied
mathematics. This includes a huge range of mathematics, including
algebra, calculus, elementary to very advanced number theory,
cryptography, numerical computation, commutative algebra, group
theory, combinatorics, graph theory, exact linear algebra and much
more.  Sage combines various software packages and seamlessly
integrates their functionality into a common experience. It is well
suited for education, studying and research.  The interface to Sage is
a notebook in a web-browser or the command-line. 
Inside the Sage notebook (see Figure~\ref{fig:sagenb}), you can create
embedded graphics, beautifully typeset mathematical expressions, add
and delete input, and share your work across the network.

\begin{figure}[ht]
\begin{center}
\includegraphics[width=0.55\textwidth]{nb2}
\caption{A Screenshot Showing Sage\label{fig:sagenb}}
\end{center}
\end{figure}

Instead of students and researchers having to pay to buy expensive
commercial mathematical software, they now have the option to use Sage
for free.  At many institutions, purchasing computer
software---especially mathematical software---is a significant burden,
and Sage has helped address this problem.  Moreover, because Sage is
free, it is available to many more undergraduates, high school
students, and non-mathematicians.

A central technical advantage of Sage over existing mathematical
software is that Sage uses the mainstream Python programming language.
Because Sage is built on Python, modern exception handling and name
spaces are inherent. Sage can be easily extended with new user-defined
data types, and since Python is such a widely used programming
language, there already exists a huge variety of extension modules.
In addition, Sage uses the Cython compiler for writing extremely fast
code. In contrast, any other major mathematics software invented its
own language, often a slow interpreted language, so that any extension
requires a new implementation.  And one must not forget that computer
based research also involves non-mathematical programming, e.g.,
create a web server to distribute the results of computations, or
connect to an online database or web page and parse the results. By
using Python, all this is easily done in Sage.


\section{Objectives}
%What is the project designed to accomplish?

The specific goals of this RRF proposal are:
\begin{enumerate}
\item {\bf Symbolic Calculus:} Improve Sage's symbolic calculus
  functionality to make Sage a much better tool for undergraduate
  education and applied mathematics.
\item {\bf The Sage Notebook:} Make the Sage notebook server more
  scalable and robust for collaboration and educational applications.
\end{enumerate}

\subsection{Symbolic Calculus}
Sage has sophisticated symbolic manipulation capabilities, which make
it very useful for teaching calculus and doing the kind of symbolic
manipulation that forms the cornerstone of classical mathematics.  Much
of this functionality was initially implemented in Sage by the UW {\em
  undergraduate} Bobby Moretti, who devoted nearly a year of hard work
to this problem in 2006.  Moretti's implementation is a ``reference
implementation'' in that it establishes how the Sage functionality for
symbolic manipulation should work, and uses an old program called
Maxima behind the scenes to make it actually work.  But this
functionality is slow, in some cases, very, very slow, especially
compared to Maple and Mathematica.

In August 2008, the PI spent nearly every waking moment during two
intense weeks modifying the Ginac C++ library so that it could do
arithmetic with Python objects and integrate well with the Sage
library.  The resulting Python library is called Pynac and is
extremely fast and robust.  Burcin Erocal, a graduate student at the
Research Institute for Symbolic Computation in Linz, Austria, has
subsequently worked part time on filling in additional functionality
needed to make this library the default library for symbolic
manipulation in Sage.

The PI proposes to spend a full six weeks working very hard to push
through full Sage/Pynac integration, thus completely replacing the
current reference implementation of symbolic manipulation in Sage.
This will dramatically improve the speed of Sage for symbolic
calculus.

\vspace{2ex}
\noindent{\bf Specific goal:} {\em Finish implementing 
  functionality so that symbolic computation in Sage uses Pynac
  instead of the current slow Maxima-based system.  This includes
  extensive support for tight integration between Pynac and the rest
  of Sage and a native implementation of the solve command in Pynac.}

\subsection{The Sage Notebook}
\begin{figure}
\begin{center}
\includegraphics[width=0.8\textwidth]{nb1}
\caption{Interactive Image Compression in the Sage Notebook\label{fig:interact}}
\end{center}
\end{figure}
The Sage notebook is an AJAX application, like Gmail or Google Maps.
It provides an interactive web-based worksheet in which one can enter
arbitrary Sage commands, see beautifully typeset output, create 2-D
and 3-D graphics, publish worksheets, and collaborate with other
users.

The PI and several UW undergraduates together developed the basic
implementation of the current version of the Sage notebook during an
extremely intense three-week coding session in Summer 2007.  This coding
work was motivated by UW's SIMUW program, which is an intense summer
mathematics program for about 25 high school students.  In 2007, the
PI taught a two-week SIMUW course using the Sage notebook on the Riemann
Hypothesis.  In 2008 he taught another 2-week SIMUW course on
Quantitative Finance using the notebook's new interactive controls
feature (see Figure~\ref{fig:interact}).


The Sage notebook is a much beloved ``killer application'' of Sage:
\begin{quote}
  In my opinion, SAGE's notebook is the real killer feature, which I
  don't recall to have seen in any other (commercial or not)
  software. I mean, this is the only scientific program that I've
  found, allowing such an easy collaborative job within local
  networks.

 -- Maurizio, the Sage mailing list.
\end{quote}

Professors at dozens of universities around the world are getting
excited about how they can leverage the Sage notebook in their
teaching.
\begin{quote}
With some colleagues in our University (Lyon, France) we have built a
project around Sage for undergraduate students... {\em And the University has
decided to support this project.} Good news.

Now we will be facing the problem to build a Sage configuration which
will work for say 200 students at the same time (students will use the
notebook), and prepare professors for Sage teaching. There are `some'
technical problems to solve...
 
-- T. Dumont, the Sage mailing list.
\end{quote}

The Sage notebook presently does not robustly scale to more than about
25 users at the same time, no matter how good the hosting hardware is.
The PI proposes to spend six weeks (supported by RRF) focused entirely
on vastly improving the robustness and scalability of the notebook.
The PI is well positioned to succeed at this project, since he is
intimately familiar with all aspects of the notebook codebase.

\vspace{2ex}
\noindent{\bf Specific goal:} {\em Improve the notebook so that it
  will robustly handle up to 200 simultaneous users when running on a
  single high-end server, as demonstrated by a robust automated test
  suite.  Implement management tools so administrators can manage the
  notebook load and better balance resources.}

\section{Procedure}

% With what methods, materials, or tools will the objectives be met?
% If access to a particular location or institution is required for
% research or data collection, state whether permission has been
% obtained.

The PI has secured funding from the NSF, PIMS, and the Center for
Communications Research to run several Sage Days workshops each year.
Much of the planning and collaboration on this project will occur
during those {\em extremely intense} workshops. Also, the PI will
collaborate with Burcin Erocal about symbolic computation, and with
numerous graduate and undergraduate students at UW who will
participate in this project.

All Sage developers receive accounts on the high-end NSF-funded {\tt
  sage.math} compute cluster in the UW mathematics department, which
greatly facilitates collaboration.

All code that comes out of this project will be made freely available
as part of Sage and will be peer reviewed as part of the standard peer
review process that {\em all} new code that gets included with Sage
goes through.

\section{Time Schedule}
% Provide a schedule showing how the proposed research can be
% accomplished during the desired support period. The support period
% is normally limited to one year. A no-cost extension of up to one
% year may be granted if requested and adequately justified.

The PI intends to work full time for three months on this project.

\begin{enumerate}
\item Weeks 1--6: Symbolic Calculus
\item Weeks 7--12: The Sage Notebook
\end{enumerate}


\section{Need for RRF Support}
% What other efforts have been made to find support for the project?
% How could the results of the work lead to further outside funding or
% commercial applications? How does this project address the mission
% of the Royalty Research Fund? For RRF Scholar applicants, provide
% documentation of teaching load (quarter, course number, title, and
% credits) in this section.

Though the PI has obtained funding for the Sage project from
Microsoft, Google, the National Science Foundation, prize money,
private donations, and contracts with industry for Sage development,
this has all been limited seed money.  The more Sage improves and
grows in quality to equal and exceed the commercial offerings, the
{\em easier} it becomes to obtain further outside funding and for Sage
to be useful in potentially thousands of educational and commercial
applications.

The goals of the current proposal are to greatly enhance Sage's level
of {\em technical credibility} in two areas that will have a huge
direct impact: Symbolic Calculus and the web-based Sage Notebook.
This project thus directly addresses the mission of the Royalty
Research Fund by providing a unique opportunity to increase the PI's
competitiveness for subsequent funding.

These two proposed goals particularly make Sage attractive to the {\em
  educational market}, a huge area of potential funding that Sage has
not had success in yet, probably because these two areas were not
developed enough to be easy and powerful enough for many educators.
Improving the notebook and symbolic calculus are unique projects that
make Sage much, much more attractive for educational funding.

\end{document}
